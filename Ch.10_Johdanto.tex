\chapter{Johdanto\label{intro}}
Kvanttitietokoneiden rakentamisessa on edistytty tasaisesti viime vuosina. Suuren kokoluokan kvanttitietokoneiden yleistyminen uhkaa tietoturvaa niiden tuoman suuren laskentatehon takia. Julkisen avaimen salaukseen ja digitaaliseen allekirjoitukseen liittyvät salausjärjestelmät ovat uhattuna muuttua hyödyttömiksi kvanttitietokoneiden myötä. Erityisesti lukuteorian ongelmiin kuten tekijöiden jakamiseen, diskreetteihin logaritmeihin ja elliptisen käyrän salaukseen perustuvat salausjärjestelmät ovat eniten uhattuna. Symmetriset salausjärjestelmät kuten lohkosalaus ja tiivistefunktiot ovat hieman lievemmin uhattuna (\cite{alagic2020status}).

Nykypäivänä salausjärjestelmiin törmää jokainen tietokoneen käyttäjä päivittäin. Salausjärjestelmillä salataan muun muassa verkon yli tapahtuvaa tietoliikennettä esimerkiksi kirjautuessa eri palveluihin kuten sähköpostiin. Esimerkiksi TLS-protokolla, jolla salataan HTTP-yhteys, käyttää yhteyden salaamiseen julkisen avaimen salausta. 

Kvanttitietokoneiden tulemisen myötä on alettu tutkia kvanttiturvallista salausta. Kvanttiturvallisella salauksella tarkoitetaan salausjärjestelmiä, jotka ovat turvallisia klassisilla tietokoneilla sekä kvanttitietokoneilla tehtyjä hyökkäyksiä vastaan ja jotka voidaan ottaa käyttöön ilman suurempia muutoksia nykyisissä protokollissa (\cite{alagic2020status}). Tutkielmassa käymme aluksi läpi hieman taustatietoja. Tutustumme kvanttitietokoneisiin ja kvanttilaskentaan ja miksi ne uhkaavat salausjärjestelmiä. Seuraavaksi tutustumme nykyaikaisiin salausjärjestelmiin. Kappaleessa kolme pääsemme itse aiheeseen, jossa käsittelemme kvanttiturvallista salausta. Perehdymme erilaisiin kvanttiturvallisiin salausjärjestelmiin.