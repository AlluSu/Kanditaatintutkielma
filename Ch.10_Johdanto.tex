\chapter{Johdanto\label{intro}}
Kvanttitietokoneiden rakentamisessa on edistytty tasaisesti viime vuosina (\cite{alagic2020status}). Suuren kokoluokan kvanttitietokoneiden yleistyminen uhkaa tietoturvaa niiden tuoman suuren laskentatehon takia. Etenkin nykyiset julkisen avaimen salaukseen liittyvät salausjärjestelmät ovat uhattuna muuttua turvattomiksi kvanttitietokoneiden myötä.

Monet julkisen avaimen salausjärjestelmät perustuvat matemaattisiin ongelmiin, jotka on helppo laskea yhteen suuntaan, mutta niiden peruuttaminen tai käänteisoperaatio on taas vaikeaa. Esimerkiksi RSA-algoritmi hyödyntää salauksessaan suurten kokonaislukujen tekijöihinjakoa. Tämä on nykypäivän tietokoneilla vaikea laskennallinen ongelma jos luku on alkuluku ja tarpeeksi suuri.

Julkisen avaimen salauksen suurin uhka on Shorin algoritmi. Peter Shor kehitti vuonna 1994 algoritmin, jonka avulla suurten kokonaislukujen tekijöihinjako voidaan laskea tehokkaasti. Shorin algoritmi on kvanttialgoritmi, joka käyttää kvanttirinnakkaisuutta apunaan.

Kvanttitietokoneiden tulemisen myötä on alettu tutkia kvanttiturvallista salausta (\emph{post-quantum cryptography}). Kvanttiturvallisella salauksella tarkoitetaan salausjärjestelmiä, jotka ovat turvallisia klassisilla tietokoneilla sekä kvanttitietokoneilla tehtyjä hyökkäyksiä vastaan ja jotka voidaan ottaa käyttöön nykyisten salausjärjestelmien tilalle ilman suurempia muutoksia esimerkiksi nykyisessä tietoliikenteessä ja eri tietoliikenneprotokollissa.

Kvanttiturvallisia salausmenetelmiä on useita ja vanhimmat ideat ovat peräisin 1970-luvulta. Kvanttiturvallisten salausmenetelmien tarjoama suojaus perustuu vaikeisiin matemaattisiin ongelmiin, jotka kuuluvat NP-koviin (\emph{NP-hard}) ongelmiin. Lisäksi ongelmiin ei tunneta olemassaolevia kvanttialgoritmeja.

Tutkielmassa käymme aluksi läpi taustatietoja aiheeseen liittyen. Tutustumme ensin nykyisiin salausjärjestelmiin, jonka jälkeen tutustumme omassa kappaleessaan kvanttitietokoneisiin ja kvanttilaskentaan. Kappaleessa 4 pääsemme itse aiheeseen, jossa käsittelemme kvanttiturvallista salausta ja perehdymme erilaisiin kvanttiturvallisiin salausmenetelmiin. Lopuksi on yhteenveto, jossa käsittelemme tutkimusalueen eli kvanttiturvallisen salauksen nykytilaa ja tulevaisuuden näkymiä.