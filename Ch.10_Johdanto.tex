\chapter{Johdanto\label{intro}}
Kvanttitietokoneiden rakentamisessa on edistytty tasaisesti viime vuosina (\cite{alagic2020status}). Suuren kokoluokan kvanttitietokoneiden yleistyminen uhkaa tietoturvaa niiden tuoman suuren laskentatehon takia. Julkisen avaimen salaukseen ja digitaaliseen allekirjoitukseen liittyvät salausjärjestelmät ovat uhattuna muuttua hyödyttömiksi kvanttitietokoneiden myötä. Erityisesti lukuteorian ongelmiin kuten tekijöiden jakamiseen, diskreetteihin logaritmeihin ja elliptisen käyrän salaukseen perustuvat salausjärjestelmät ovat eniten uhattuna. Symmetriset salausjärjestelmät kuten lohkosalaus ja tiivistefunktiot ovat hieman lievemmin uhattuna.

Kvanttitietokoneiden tulemisen myötä on alettu tutkia kvanttiturvallista salausta (\cite{alagic2020status}). Kvanttiturvallisella salauksella tarkoitetaan salausjärjestelmiä, jotka ovat turvallisia klassisilla tietokoneilla sekä kvanttitietokoneilla tehtyjä hyökkäyksiä vastaan ja jotka voidaan ottaa käyttöön nykyisten salausjärjestelmien tilalle ilman suurempia muutoksia nykyisissä  järjestelmissä esimerkiksi eri tietoliikenneprotokollissa.

Tutkielmassa käymme aluksi läpi hieman taustatietoja. Tutustumme ensin nykyisiin salausjärjestelmiin ja sitten kvanttitietokoneisiin. Kappaleessa kolme pääsemme itse aiheeseen, jossa käsittelemme kvanttiturvallista salausta. Perehdymme erilaisiin kvanttiturvallisiin salausjärjestelmiin.