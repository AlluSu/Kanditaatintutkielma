\chapter{Johdanto\label{intro}}
Kvanttitietokoneiden rakentamisessa on edistytty tasaisesti viime vuosina (\cite{alagic2020status}). Suuren kokoluokan kvanttitietokoneiden yleistyminen uhkaa tietoturvaa niiden tuoman suuren laskentatehon takia. Etenkin nykyiset julkisen avaimen salaukseen liittyvät salausjärjestelmät ovat uhattuna muuttua turvattomiksi kvanttitietokoneiden myötä.

Julkisen avaimen salaus on laajalti nykypäivänä käytössä oleva salausjärjestelmä, jolla salataan muun muassa tietoliikennettä. Esimerkiksi TLS-protokollan suojaus, joka suojaa miljardeja verkkoyhteyksiä päivittäin, perustuu julkisen avaimen salaukseen (\cite{buchmann2016post}). %oma viite, yleisesti tiedossa oleva asia
Monet julkisen avaimen salausjärjestelmät perustuvat matemaattisiin ongelmiin, jotka on helppo laskea yhteen suuntaan, mutta niiden peruuttaminen tai käänteisoperaatio on taas vaikeaa (\cite{mavroeidis2018impact}). Esimerkki tällaisesta ongelmasta on suurten kokonaislukujen tekijöihinjako. Suurten kokonaislukujen tekijöihinjako on vaikea laskennallinen ongelma ja sille ei tunneta tehokkaita eli polynomisessa ajassa ratkeavia ratkaisuja nykyisillä tietokoneilla (\cite{doi:10.1137/S0036144598347011}). Vaikealla tarkoitetaan tässä tapauksessa sitä, että ongelman ratkeamista joudutaan odottamaan niin kauan, että lopputulosta ei välttämättä saada järkevässä ajassa tai tiedolla ei ole enää merkitystä kun laskenta on loppunut. %ehkä viite
RSA on yksi ensimmäisiä ja tänäkin päivänä käytetyimpiä julkisen avaimen salausjärjestelmiä, jonka salaus perustuu suurten kokonaislukujen tekijöihinjakoon (\cite{montgomery1994survey}).

Erityisesti Shorin algoritmi uhkaa muuttaa julkisen avaimen salauksen turvattomaksi (\cite{mavroeidis2018impact}). Vuonna 1994 matemaatikko Peter Shor esitteli kvanttialgoritmin, jolla julkisen avaimen salaukseen perustuvat vaikeat matemaattiset ongelmat, kuten suurten kokonaislukujen tekijöihinjako, voidaan ratkaista tehokkaasti eli polynomisessa ajassa kvanttitietokoneella (\cite{doi:10.1137/S0036144598347011}).

Kvanttitietokoneiden tulemisen myötä on alettu tutkia kvanttiturvallista salausta (\emph{post-quantum cryptography}) (\cite{alagic2020status}). Kvanttiturvallisella salauksella tarkoitetaan salausjärjestelmiä, jotka ovat turvallisia klassisilla tietokoneilla sekä kvanttitietokoneilla tehtyjä hyökkäyksiä vastaan ja jotka voidaan ottaa käyttöön nykyisten salausjärjestelmien tilalle ilman suurempia muutoksia esimerkiksi nykyisessä tietoliikenteessä ja eri tietoliikenneprotokollissa.

Erilaisia kvanttiturvallisia salausmenetelmiä on tutkittu useita (\cite{mavroeidis2018impact}) ja vanhimmat ideat ovat peräisin 1970-luvulta (\cite{repka2014overview}). Monet kvanttiturvalliset salausjärjestelmät perustuvat laskennallisesti vaativiin ongelmiin, jotka ovat vaikeita ratkaista klassisilla tietokoneilla että kvanttitietokoneilla. %oma juttu
Lisäksi nämä laskennalliset ongelmat ovat luonteeltaan sellaisia, että niihin ei ole osattu soveltaa Shorin algoritmia (\cite{bernstein2017post}).
Buchmann et al. (2016) mukaan salausjärjestelmä on kvanttiturvallinen, jos se perustuu laskennalliseen ongelmaan joka on mahdoton ratkaista klassisella tietokoneella ja ei ratkea polynomisessa ajassa kvanttitietokoneella. National Institute of Standards and Technology (NIST) ylläpitää kvanttiturvallisten julkisen avaimen salausjärjestelmien standardointiprosessia (\cite{alagic2020status}). Kirjoittamishetkellä on käynnissä standardointiprosessin kolmas kierros, johon on valittu seitsemän finalistia. 

Tutkielmassa käsittelemme kolme kvanttiturvallista salausmenetelmää; koodipohjainen, hilapohjainen ja monimuuttujapohjainen salaus. Tutkielmassa esitellään näiden kvanttiturvallisuutta ja yleisiä ominiaisuuksia esimerkiksi julkisen avaimen koossa ja suorituskyvyssä.

Tutkielmassa käymme aluksi läpi taustatietoja aiheeseen liittyen. Tutustumme ensin nykyisiin salausjärjestelmiin jonka jälkeen tutustumme kvanttitietokoneisiin, kvanttilaskentaan sekä Shorin algoritmiin. Samassa yhteydessä käsittelemme nykypäivän salausjärjestelmien kvanttiturvallisuutta. Seuraavaksi pääsemme itse aiheeseen, jossa käsittelemme kvanttiturvallista salausta ja perehdymme erilaisiin kvanttiturvallisiin salausmenetelmiin sekä niitä hyödyntäviin julkisen avaimen salausjärjestelmiin. Lopuksi on yhteenveto, jossa käsittelemme tutkielmassa tehtyjä havaintoja liittyen kvanttiturvallisiin salausjärjestelmiin.