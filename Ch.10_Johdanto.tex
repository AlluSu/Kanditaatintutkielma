\chapter{Johdanto\label{intro}}
Kvanttitietokoneiden rakentamisessa on edistytty tasaisesti viime vuosina (\cite{alagic2020status}). Suuren kokoluokan kvanttitietokoneiden yleistyminen uhkaa tietoturvaa niiden tuoman suuren laskentatehon takia. Etenkin nykyiset julkisen avaimen salaukseen liittyvät salausjärjestelmät ovat uhattuna muuttua turvattomiksi kvanttitietokoneiden myötä.

Monet julkisen avaimen salausjärjestelmät perustuvat matemaattisiin ongelmiin, jotka on helppo laskea yhteen suuntaan, mutta niiden peruuttaminen tai käänteisoperaatio on taas vaikeaa (\cite{mavroeidis2018impact}). Esimerkki tällaisesta ongelmasta on suurten kokonaislukujen tekijöihinjako. Suurten kokonaislukujen tekijöihinjako on vaikea laskennallinen ongelma ja sille ei tunneta tehokkaita eli polynomisessa ajassa ratkeavia ratkaisuja (\cite{doi:10.1137/S0036144598347011}). Esimerkiksi RSA-algoritmi hyödyntää salauksessaan suurten kokonaislukujen tekijöihinjakoa (\cite{montgomery1994survey}).

Shorin algoritmi uhkaa muuttaa julkisen avaimen salauksen turvattomaksi (\cite{mavroeidis2018impact}). Vuonna 1994 matemaatikko Peter Shor esitteli kvanttialgoritmin, jolla julkisen avaimen salaukseen perustuvat vaikeat matemaattiset ongelmat, kuten suurten kokonaislukujen tekijöihinjako, voidaan ratkaista tehokkaasti eli polynomisessa ajassa kvanttitietokoneella. 

Kvanttitietokoneiden tulemisen myötä on alettu tutkia kvanttiturvallista salausta (\emph{post-quantum cryptography}) (\cite{alagic2020status}). Kvanttiturvallisella salauksella tarkoitetaan salausjärjestelmiä, jotka ovat turvallisia klassisilla tietokoneilla sekä kvanttitietokoneilla tehtyjä hyökkäyksiä vastaan ja jotka voidaan ottaa käyttöön nykyisten salausjärjestelmien tilalle ilman suurempia muutoksia esimerkiksi nykyisessä tietoliikenteessä ja eri tietoliikenneprotokollissa.

Erilaisia kvanttiturvallisia salausmenetelmiä on tutkittu useita (\cite{mavroeidis2018impact}) ja vanhimmat ideat ovat peräisin 1970-luvulta (\cite{repka2014overview}). Monet kvanttiturvalliset salausjärjestelmät perustuvat ongelmiin, jotka ovat NP-kovia klassisella tietokoneella ja siten myös vaikeita ongelmia suorittaa kvanttitietokoneella. Lisäksi näihin kvanttiturvallisiin salausmenetelmiin ei ole osattu soveltaa Shorin algoritmia (\cite{bernstein2017post}). \emph{National Institute of Standards and Technology} eli NIST ylläpitää kvanttiturvallisten julkisen avaimen salausjärjestelmien standardointiprosessia (\cite{alagic2020status}). Kirjoittamishetkellä on käynnissä standardointiprosessin kolmas kierros, jossa on valittu seitsemän finalistia.

Tutkielmassa käymme aluksi läpi taustatietoja aiheeseen liittyen. Tutustumme ensin nykyisiin salausjärjestelmiin, jonka jälkeen tutustumme omassa kappaleessaan kvanttitietokoneisiin ja kvanttilaskentaan. Seuraavaksi pääsemme itse aiheeseen, jossa käsittelemme kvanttiturvallista salausta ja perehdymme erilaisiin kvanttiturvallisiin salausmenetelmiin. Lopuksi on yhteenveto, jossa käsittelemme tutkimusalueen eli kvanttiturvallisen salauksen nykytilaa ja tulevaisuuden näkymiä.