\chapter{Johdanto\label{intro}}
Kvanttitietokoneiden rakentamisessa on edistytty tasaisesti viime vuosina (\cite{alagic2020status}). Suuren kokoluokan kvanttitietokoneiden yleistyminen uhkaa tietoturvaa niiden tuoman suuren laskentatehon takia. Etenkin nykyiset julkisen avaimen salaukseen liittyvät salausjärjestelmät ovat uhattuna muuttua turvattomiksi kvanttitietokoneiden myötä.

Julkisen avaimen salaus on laajalti nykypäivänä käytössä oleva salausjärjestelmä, jolla salataan muun muassa tietoliikennettä. %oma viite, yleisesti tiedossa oleva asia
Monet julkisen avaimen salausjärjestelmät perustuvat matemaattisiin ongelmiin, jotka on helppo laskea yhteen suuntaan, mutta niiden peruuttaminen tai käänteisoperaatio on taas vaikeaa (\cite{mavroeidis2018impact}). Esimerkki tällaisesta ongelmasta on suurten kokonaislukujen tekijöihinjako. Suurten kokonaislukujen tekijöihinjako on vaikea laskennallinen ongelma ja sille ei tunneta tehokkaita eli polynomisessa ajassa ratkeavia ratkaisuja nykyisillä tietokoneilla (\cite{doi:10.1137/S0036144598347011}). Vaikealla tarkoitetaan tässä tapauksessa sitä, että ongelman ratkeamista joudutaan odottamaan niin kauan, että lopputulosta ei välttämättä saada missään äärellisessä ajassa. Esimerkiksi salauksen purkamisessa kestää niin kauan, että tiedolla ei ole enää merkitystä kun laskenta on loppunut. %ehkä viite
RSA on yksi ensimmäisiä ja käytetyimpiä julkisen avaimen salausjärjestelmiä, joka perustuu suurten kokonaislukujen tekijöihinjakoon (\cite{montgomery1994survey}).

Erityisesti Shorin algoritmi uhkaa muuttaa julkisen avaimen salauksen turvattomaksi (\cite{mavroeidis2018impact}). Vuonna 1994 matemaatikko Peter Shor esitteli kvanttialgoritmin, jolla julkisen avaimen salaukseen perustuvat vaikeat matemaattiset ongelmat, kuten suurten kokonaislukujen tekijöihinjako, voidaan ratkaista tehokkaasti eli polynomisessa ajassa kvanttitietokoneella (\cite{doi:10.1137/S0036144598347011}).

Kvanttitietokoneiden tulemisen myötä on alettu tutkia kvanttiturvallista salausta (\emph{post-quantum cryptography}) (\cite{alagic2020status}). Kvanttiturvallisella salauksella tarkoitetaan salausjärjestelmiä, jotka ovat turvallisia klassisilla tietokoneilla sekä kvanttitietokoneilla tehtyjä hyökkäyksiä vastaan ja jotka voidaan ottaa käyttöön nykyisten salausjärjestelmien tilalle ilman suurempia muutoksia esimerkiksi nykyisessä tietoliikenteessä ja eri tietoliikenneprotokollissa.

Erilaisia kvanttiturvallisia salausmenetelmiä on tutkittu useita (\cite{mavroeidis2018impact}) ja vanhimmat ideat ovat peräisin 1970-luvulta (\cite{repka2014overview}). Monet kvanttiturvalliset salausjärjestelmät perustuvat laskennallisesti vaativiin ongelmiin, jotka ovat vaikeita ratkaista klassisilla tietokoneilla että kvanttitietokoneilla. %oma juttu
 Lisäksi nämä laskennalliset ongelmat ovat luonteeltaan sellaisia, että niihin ei ole osattu soveltaa Shorin algoritmia (\cite{bernstein2017post}). \emph{National Institute of Standards and Technology} eli NIST ylläpitää kvanttiturvallisten julkisen avaimen salausjärjestelmien standardointiprosessia (\cite{alagic2020status}). Kirjoittamishetkellä on käynnissä standardointiprosessin kolmas kierros, johon on valittu seitsemän finalistia. Tutkielmassa käsittelemme kolme kvanttiturvallista salausmenetelmää; koodipohjaisen, hilapohjaisen ja monimuuttujapohjaisen salauksen. Jokaisessa salausmenetelmässä on omat hyvät ja huonot puolensa ja yksikään ei osoittautunut toisistaan paremmaksi.

Tutkielmassa käymme aluksi läpi taustatietoja aiheeseen liittyen. Tutustumme ensin nykyisiin salausjärjestelmiin ja niiden kvanttiturvallisuuteen. Sen jälkeen tutustumme kvanttitietokoneisiin ja kvanttilaskentaan sekä Shorin algoritmiin. Seuraavaksi pääsemme itse aiheeseen, jossa käsittelemme kvanttiturvallista salausta ja perehdymme erilaisiin kvanttiturvallisiin salausmenetelmiin sekä niitä hyödyntäviin julkisen avaimen salausjärjestelmiin. Lopuksi on yhteenveto, jossa käsittelemme tutkimusalueen eli kvanttiturvallisen salauksen nykytilaa ja tulevaisuuden näkymiä.