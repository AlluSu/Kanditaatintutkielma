\chapter{Johdanto\label{intro}}
Kvanttitietokoneiden rakentamisessa on edistytty tasaisesti viime vuosina (\cite{alagic2020status}). Suuren kokoluokan kvanttitietokoneiden yleistyminen uhkaa tietoturvaa niiden tuoman suuren laskentatehon takia. Etenkin julkisen avaimen salaukseen liittyvät salausjärjestelmät ovat uhattuna muuttua hyödyttömiksi kvanttitietokoneiden myötä. Tiivistefunktiot ja symmetriset salausjärjestelmät kuten lohkosalaus ovat lievemmin uhattuna.

Kvanttitietokoneiden tulemisen myötä on alettu tutkia kvanttiturvallista salausta (\emph{post-quantum cryptography}). Kvanttiturvallisella salauksella tarkoitetaan salausjärjestelmiä, jotka ovat turvallisia klassisilla tietokoneilla sekä kvanttitietokoneilla tehtyjä hyökkäyksiä vastaan ja jotka voidaan ottaa käyttöön nykyisten salausjärjestelmien tilalle ilman suurempia muutoksia esimerkiksi nykyisessä tietoliikenteessä ja eri tietoliikenneprotokollissa.

Tutkielmassa käymme aluksi läpi hieman taustatietoja. Tutustumme ensin nykyisiin salausjärjestelmiin ja sitten kvanttitietokoneisiin ja kvanttilaskentaan. Kappaleessa kolme pääsemme itse aiheeseen, jossa käsittelemme kvanttiturvallista salausta. Perehdymme erilaisiin kvanttiturvallisiin salausjärjestelmiin.