\chapter{Kvanttilaskennasta ja salausjärjestelmistä\label{methods}}

\section{Kvanttilaskenta ja kvanttitietokoneet}
Klassisessa eli niin sanotussa tavallisessa tietokoneessa informaatio esitetään bitteinä ja yksi bitti on tiedon perusyksikkö. Yksittäinen bitti voi saada vain yhden diskreetin arvon kerrallaan ja näitä arvoja kuvataan yleensä arvoilla 0 ja 1. Bitti voi siis olla arvoltaan vuorostaan joko 0 tai 1.

Kvanttitietokoneen laskennan perusyksikkö taas on kubitti eli kvanttibitti. Kubitti voi olla bitin tavoin kahdessa eri tilassa. Merkitään näitä tiloja symboleilla
$\ket{0}$ ja $\ket{1}$. Kvanttimekaniikassa notaatiota $\ket{}$ kutsutaan tilaksi tai vektoriksi. Kubitti eroaa tavallisesta bitistä siten, että se voi olla tilojen $\ket{0}$ tai $\ket{1}$ sijaan niin sanotussa superpositiotilassa. Tämä superpositiotila on taas tilojen $\ket{0}$ ja $\ket{1}$ lineaarikombinaatio. Tämä tarkoittaa siis sitä, että mittauksen aikana kubitti voi olla arvoltaan 0 tai 1, mutta laskennan aikana se on yhtäaikaa sekä 0 että 1. Eli tilat $\ket{0}$ ja $\ket{1}$ voivat esiintyä yhtäaikaa.

\section{Nykyiset salausjärjestelmät}
Kryptografia on matemaattisten systeemien tutkimista johon kuuluu kahdenlaisia suojausongelmia, jotka ovat yksityisyys ja autentikointi. Nykypäivänä eli niin sanotun modernin kryptografian aikana salausjärjestelmät ovat arkipäivää. Esimerkiksi sähköpostitilille kirjautuminen kysyy salasanaa. Yleensä tämä salasana lähetetään verkon yli takaisin kirjautuessa sähköpostipalvelimelle salattuna käyttäen jotain salausalgoritmia. Näin esimerkiksi verkkoliikennettä skannaava pahantahtoinen taho ei saa selville salasanaa kovin helposti muun tietoliikenteen joukosta. Lisäksi palvelimelle lähetetty salasana on todennäköisesti tallennettu johonkin tietokantaan salattuna käyttäen jotain toista algoritmia. Tällöin tietokannan joutuessa pahantahtoisen tahon käsiin salasanojen salauksen purkaminen hidastaa tahon toimia.

\subsection{Symmetrinen salaus}
 Symmetrisessä salauksessa tai salaisen avaimen salauksessa sekä viestin lähettäjän eli salaajan sekä viestin vastaanottajan eli salauksen purkajan täytyy tietää avain. Toisin sanoen symmetrisessä salauksessa salaus puretaan samalla avaimella kuin millä se salataan. Esimerkiksi lohkosalaus, englanniksi block cipher, on symmetrinen salausjärjestelmä. Symmetrisessä salauksessa ongelmallista on salausavaimen jakaminen osapuolien välillä. Jos symmetriseen salaukseen käytetty salausavain joutuu jonkun kolmannen osapuolen käsiin, voi tämä osapuoli purkaa salauksen tällä avaimella.
 
 \subsection{Julkisen avaimen salaus}
 Vuonna 1976 Diffie ja Hellman esittelivät artikkelissaan "New directions in cryptography" epäsymmetrisen eli julkisen avaimen salauksen. Epäsymmetrisessa eli julkisen avaimen salauksessa on käytössä kaksi avainta. Toinen on niin kutsuttu julkinen avain, englanniksi public key, ja toinen salainen avain, englanniksi private key. Julkisen avaimen salausjärjestelmässä jokainen käyttäjä luo itselleen parin avaimia eli salausavaimen ja purkuavaimen. Tässä tapauksessa purkuavain on salainen avain ja sitä ei missään nimessä pidä kertoa muille. Tällöin salausavain on julkinen avain. Viestin lähetys toimii nyt siten, että lähettäjä S lähettää viestin vastaanottajalle R. S salaa viestin R:n julkisella avaimella. R saa S:n viestin vastaan ja purkaa sen salauksen omalla yksityisellä avaimellaan. Ero symmetriseen salaukseen epäsymmetrisessa salauksessa on siis se, että salaus puretaan eri avaimella kuin millä se salataan.
 
 Tunnetuin julkisen avaimen salausalgoritmi on RSA-algoritmi jonka julkaisivat vuonna 1978 Ron Rivest, Adi Shamir ja Leonard Adleman. RSA-algoritmi perustuu Diffien ja Hellmanin artikkeliin. Näiden salausjärjestelmien tarjoama suojaus perustuu tiettyihin vaikeisiin lukuteorian ongelmiin. Näitä ongelmia ovat muun muassa suurten alkulukujen jako tekijöihinsä (Integer Factorization Problem) sekä niin kutsuttu diskreetin logaritmin ongelma (Discrete Logarithm Problem). Esimerkiksi RSA:n tarjoama suojaus perustuu suurten alkulukujen tekijöiden jakoon. Nämä edellä mainitut ongelmat ovat laskennallisesti hyvin haastavia ja niihin ei tunneta tehokkaita ratkaisuja isoilla syötteillä klassisilla tietokoneilla. Siksi niitä on hyvä käyttää salausjärjestelmissä. Toisaalta kummallekkin ongelmalle tunnetaan tehokkaita ratkaisuja, jotka voidaan suorittaa kvanttitietokoneella. Peter Shor esitti kuuluisan Shorin algoritminsa, jolla nämä voidaan ratkaista vuonna 1994.
 
 \subsection{Tiivistefunktiot}
 Tiivistefunktiot eli hashit (englanniksi hash-functions) ovat funktioita, jotka kuvaavat ison joukon viestejä pieneksi joukoksi tiivisteitä. Toisin sanoen, tiivistefunktio ottaa viestin M ja muodostaa siitä aina yhtäpitkän tiivisteen MD. Pointtina tiivistefunktioissa on se, että ne toimivat aina yhteen suuntaan. Viestistä M on helposti saatavissa tiiviste MD, mutta tiivisteestä MD ei voi helposti päätellä viestiä M. Tiivistefunktioita on monia erilaisia, mutta emmme käy tässä kaikkia läpi, koska niiden yksityiskohdat eivät ole relevantteja kvanttiturvallisten salausjärjestelmien kanssa.
 
 \subsection{Digitaaliset allekirjoitukset}
 Digitaaliset allekirjoitukset eli digital signatures.
