\chapter{Kvanttilaskennasta ja salausjärjestelmistä\label{methods}}

\section{Kvanttilaskenta ja kvanttitietokoneet}
Klassisessa eli niin sanotussa tavallisessa tietokoneessa informaatio esitetään bitteinä ja yksi bitti on tiedon perusyksikkö. Yksittäinen bitti voi saada vain yhden diskreetin arvon kerrallaan ja näitä arvoja kuvataan yleensä arvoilla 0 ja 1. Bitti voi siis olla arvoltaan vuorostaan joko 0 tai 1. Klassisessa tietokoneessa näitä bittejä käsitellään loogisten porttien kautta.

Kvanttitietokoneen laskennan perusyksikkö taas on kubitti eli kvanttibitti. Kubitti voi olla bitin tavoin myös kahdessa eri tilassa. Merkitään näitä tiloja symboleilla
$\ket{0}$ ja $\ket{1}$. \footnote{Kvanttimekaniikassa notaatiota $\ket{}$ kutsutaan Diracin notaatioksi. Se on yleinen notaatio kvanttimekaniikassa tarkoittaen tilaa.} Kubitti eroaa tavallisesta bitistä siten, että se voi olla tilojen $\ket{0}$ tai $\ket{1}$ sijaan niin sanotussa superpositiotilassa. Tämä tarkoittaa siis sitä, että mittauksen aikana kubitti voi olla arvoltaan 0 tai 1, mutta laskennan aikana se on yhtäaikaa sekä 0 että 1. Eli tilat $\ket{0}$ ja $\ket{1}$ voivat esiintyä yhtäaikaa. Kvanttitietokoneessa kubitteja käsittelee kvanttiportit kuten klassisessa tietokoneessa bittejä käsittelee loogiset portit.

Esimerkiksi kolmella kubitilla laskentaa tekevä kvanttitietokone mahdollistaa 8 eri tilaa samanaikaisesti. Kolmella bitillä laskentaa tekevä klassinen tietokone mahdollistaa myös 8 eri tilaa, mutta vain yksi näistä kahdeksasta tilasta voi olla kerrallaan voimassa. Kvanttitietokoneen tehokkaamman laskentatehon mahdollistaa siis mahdollisuus suorittaa laskenta samanaikaisesti eli rinnakkain jokaiselle kubitin eri tilalle. Kvanttitietokone tarjoaa siis eksponentiaalisesti nopeamman laskentatehon, joka uhkaa muun muassa julkisen avaimen salausta.

\section{Nykyiset salausjärjestelmät}
Kryptografia on matemaattisten systeemien tutkimista, jonka tarkoituksena on suojata yksityisyyttä ja mahdollistaa autentikointi. Nykypäivänä eli niin sanotun modernin kryptografian aikana salausjärjestelmät ovat arkipäivää. Esimerkiksi sähköpostitilille kirjautuminen kysyy salasanaa. Nämä kirjautumistiedot lähetetään salattuna verkon yli sähköpostipalvelimelle käyttäen jotain salausjärjestelmää. Näin esimerkiksi verkkoliikennettä tutkiva kolmas osapuoli ei saa selville kirjautumistietoja kovin helposti muun tietoliikenteen joukosta. Lisäksi palvelimelle lähetetty salasana on todennäköisesti tallennettu johonkin sähköpostipalvelimen tietokantaan salattuna esimerkiksi tiivisteenä käyttäen jotain tiivistefunktiota. Seuraavaksi käsittelemme nykyaikaisia salausjärjestelmiä ja menetelmiä.

\subsection{Symmetrinen salaus}
 Symmetrisessä salauksessa sekä viestin lähettäjän eli salaajan sekä viestin vastaanottajan eli salauksen purkajan täytyy tietää sama avain. Toisin sanoen symmetrisessä salauksessa viestin salaus puretaan samalla avaimella kuin millä viesti salataan.  Symmetrisessä salauksessa ongelmallista on salausavaimen jakaminen osapuolien välillä. Jos symmetriseen salaukseen käytetty salausavain joutuu jonkun kolmannen osapuolen käsiin, voi tämä osapuoli purkaa salauksen tällä avaimella. Lohkosalaus (\emph{block cipher}) on esimerkki symmetrisestä salausjärjestelmästä.
 
 \subsection{Julkisen avaimen salaus}
 Vuonna 1976 Diffie ja Hellman esittelivät artikkelissaan \emph{New directions in cryptography} epäsymmetrisen eli julkisen avaimen salauksen. Epäsymmetrisessa eli julkisen avaimen salauksessa on käytössä kaksi avainta. Toinen on niin kutsuttu julkinen avain (\emph{public key}) ja toinen salainen avain (\emph{private key}). Julkisen avaimen salausjärjestelmässä jokainen käyttäjä luo itselleen parin avaimia, jotka ovat salausavain ja purkuavain. Tässä tapauksessa purkuavain on salainen avain ja sitä ei missään nimessä pidä paljastaa muille. Tällöin salausavain on julkinen avain, joka saa olla julkisesti esillä. Viestin lähetys toimii nyt siten, että lähettäjä \emph{S} lähettää viestin vastaanottajalle \emph{R}. \emph{S} salaa viestin \emph{R}:n julkisella avaimella. \emph{R} saa \emph{S}:n viestin vastaan ja purkaa sen salauksen omalla yksityisellä avaimellaan. Ero symmetriseen salaukseen epäsymmetrisessa salauksessa on siis se, että salaus puretaan eri avaimella kuin millä se salataan.
 
 Tunnetuin julkisen avaimen salausalgoritmi on RSA-algoritmi jonka julkaisivat vuonna 1978 Ron Rivest, Adi Shamir ja Leonard Adleman. RSA-algoritmin toiminta perustuu Diffien ja Hellmanin artikkeliin vuodelta 1976. Julkisen avaimen salausjärjestelmien tarjoama suojaus perustuu tiettyihin vaikeisiin lukuteorian ongelmiin. Näitä ongelmia ovat muun muassa suurten alkulukujen jako tekijöihinsä (\emph{Integer Factorization Problem}) sekä niin kutsuttu diskreetin logaritmin ongelma (\emph{Discrete Logarithm Problem}). Esimerkiksi RSA:n tarjoama suojaus perustuu suurten alkulukujen tekijöiden jakoon. Nämä edellä mainitut ongelmat ovat laskennallisesti hyvin haastavia ja niihin ei tunneta tehokkaita ratkaisuja isoilla syötteillä klassisilla tietokoneilla minkä takia niitä on hyvä käyttää salausjärjestelmissä. Toisaalta kummallekkin ongelmalle tunnetaan tehokkaita ratkaisuja, jotka voidaan suorittaa kvanttitietokoneella. Peter Shor esitti kuuluisan Shorin algoritminsa, jolla nämä voidaan ratkaista tehokkaasti kvanttitietokoneella vuonna 1994.
 
 \subsection{Tiivistefunktiot}
 Tiivistefunktiot eli hashit (\emph{hash-functions}) ovat funktioita, jotka ottavat mielivaltaisen pituisen merkkijonon ja muodostavat siitä tietyn mittaisen toisen merkkijonon jota kutsutaan tiivisteeksi. Voi siis ajatella, että tällainen hash ottaa merkkijonon ja tiivistää sen toiseksi merkkijonoksi, josta tulee sana tiiviste. Formaalimmin, tiivistefunktio ottaa viestin \emph{M} ja muodostaa siitä aina yhtäpitkän tiivisteen \emph{MD}. Yleisesti käytettyjä tiivistefunktoita ovat muun muassa SHA1 ja MD-5. Esimerkiksi SHA1 tiivistefunktio muodostaa syötteestä aina 160 bittiä pitkän tiivisteen. Tiivistefunktioita on olemassa montaa eri tyyppiä. Tärkeä ominaisuus tiivistefunktioissa on se, että ne toimivat yhteen suuntaan. Viestistä \emph{M} on helposti saatavissa tiiviste \emph{MD} funktiolla \emph{F}, mutta tiivisteestä \emph{MD} ja funktiosta \emph{F} ei voi päätellä tai muodostaa viestiä \emph{M}.
 
 \subsection{Digitaalinen allekirjoitus}
 Digitaalinen allekirjoitus (\emph{digital signature}) on menetelmä, jolla voidaan todentaa digitaalisia dokumentteja. Sen toimintaperiaate on analoginen perinteisiin allekirjoituksiin verrattuna. Digitaalisella allekirjoituksella todennetaan muun muassa digitaalisen dokumentin datan eheys. Eheydellä voidaan tässä kontekstissa tarkoittaa esimerkiksi dokumentin päivämäärää tai sisältämää informaatiota. Digitaalisen allekirjoituksen luominen yhdistelee edellä mainittuja salausjärjestelmiä kuten tiivistefunktioita ja julkisen tai symmetrisen avaimen salausta. Digitaalinen allekirjoitus on käytännössä bittijono, jonka jokin digitaalisen allekirjoituksen luomiseen tarkoitettu algoritmi luo. Tämä allekirjoitus muodostetaan allekirjoitettavien dokumenttien ja jonkin vain viestin lähettäjän määrittelemän tiedon mukaan. Digitaalisen allekirjoituksen luominen koostuu yleensä kolmesta eri vaiheesta; salausavaimen generointi, digitaalisen allekirjoituksen generointi ja digitaalisen allekirjoituksen varmennus. Salausavain voidaan generoida jollain edellämainituista menetelmistä kuten julkisen avaimen salauksella tai symmetrisella salauksella.