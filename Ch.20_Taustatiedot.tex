\chapter{Nykyiset salausjärjestelmät\label{methods}}

Käymme nyt läpi yleisimpiä salausjärjestelmiä, joihin jokainen tietojärjestelmien käyttäjä törmää nykypäivänä tiedostaen tai tiedostamattaan. Käymme läpi salausjärjestelmien ideat ja miten kvanttitietokoneet uhkaavat niitä.

Jatkossa avaimesta puhuttaessa tarkoitetaan tässä kontekstissa salausavainta, jolla voidaan purkaa tai salata lähetettävä viesti. Käytännössä tämä avain on \emph{k} bittiä pitkä bittijono, joka on muodostettu jonkin salausavainten luontiin tarkoitetun algoritmin avulla.

\section{Symmetrinen salaus}
 Symmetrisessä salauksessa sekä viestin lähettäjän eli salaajan sekä viestin vastaanottajan eli salauksen purkajan täytyy tietää sama avain (\cite{bellare2005introduction}). Toisin sanoen symmetrisessä salauksessa viestin salaus puretaan samalla avaimella kuin millä viesti salataan.  Symmetrisessä salauksessa ongelmallista on salausavaimen jakaminen turvallisesti ja luottamuksellisesti osapuolien välillä. Jos symmetriseen salaukseen käytetty salausavain joutuu jonkun kolmannen osapuolen käsiin, voi tämä osapuoli purkaa salauksen tällä avaimella. Esimerkki symmetrisestä salausjärjestelmästä on lohkosalaukseen (\emph{blockcipher}) perustuva AES (\emph{Advanced Encryption Standard}) (\cite{117146}). AES jakaa datan 128 bittiä pitkiin lohkoihin ja käyttää 128, 192 ja 256 bittisiä salausavaimia.
 
 Symmetrinen salaus on itsessään jo melko kvanttiturvallinen, koska niiden murtamiseen käytetään yleensä raa'an voiman (\emph{brute-force}) tekniikoita. Brute-forcella tarkoitetaan kaikkien mahdollisten vaihtoehtojen, esimerkiksi salausavaimien, systemaattista kokeilua. %ehkä viite
Toistaiseksi ei ole löydetty kvanttialgoritmeja, jotka murtaisivat symmetristä salausta käyttäviä järjestelmiä huomattavan tehokkaasti siten, että ne olisi uhattuna muuttua turvattomiksi. Kuitenkin \emph{Groverin algoritmia} hyödyntämällä on mahdollista nopeuttaa kvanttitietokoneella tehtyjä brute-force hyökkäyksiä. (\cite{mavroeidis2018impact}.)
 
 Lyhyesti Groverin algoritmi on kvanttialgoritmi, jota käytetään tiedon hakemiseen (\cite{hayward2008quantum}). Groverin algoritmi löytää \emph{N}-kokoisesta järjestämättömästä listasta halutun alkion $\sqrt{N}$ askeleella. Klassisilla tietokoneilla ja algoritmeilla tällaisessa tehtävässä menee aina vähintään $N$ askelta. Siten Groverin algoritmi tuo neliöllisen nopeutuksen kvanttitietokoneella tehtyihin brute-force hyökkäyksiin (\cite{mavroeidis2018impact}). Kvanttitietokone joka käyttää Groverin algoritmia \emph{n}-bittisen salausavaimen purkamiseen laskee avaimen $\sqrt{2^{n}} = 2^{n/2}$ operaatiolla. Tällöin esimerkiksi 128-bittinen AES-salaus antaisi 64-bittisen suojauksen. Toisaalta näitä hyökkäyksiä voidaan hidastaa esimerkiksi tuplaamalla salausavaimen käyttämä bittien määrä. Tällöin vaadittavien kokeilujen määrä kasvaa eksponentiaalisesti. Lohkosalaukseen perustuvia salausjärjestelmiä AES-192 ja AES-256 pidetään kvanttiturvallisina nyt ja jatkossa (\cite{mavroeidis2018impact}). Nykypäivänä 80-bittistä suojausta pidetään turvallisena rajana klassisella tietokoneella.
 
 \section{Tiivistefunktiot}
 Tiivistefunktiot (\emph{hash functions}) eli puhekielessä hashit  ovat funktioita, jotka ottavat mielivaltaisen pituisen merkkijonon \emph{M} ja muodostavat siitä aina tietyn mittaisen toisen merkkijonon \emph{MD}, jota kutsutaan tiivisteeksi. Formaalimmin tiivistefunktio on funktio
 \[h: D \to R, \text{ missä } D = \big\{0,1\big\}^{*}, R = \big\{0,1\big\}^{n}, \text{ jollain } n \geq 1.\] Voi siis ajatella, että tällainen hash ottaa merkkijonon ja tiivistää sen toiseksi merkkijonoksi, josta tulee sana tiiviste. (\cite{sobti2012cryptographic}.)
 
 Yleisesti käytettyjä tiivistefunktoita ovat muun muassa SHA1 ja MD-5. Esimerkiksi SHA1 tiivistefunktio muodostaa syötteestä aina 160 bittiä pitkän tiivisteen (\cite{bellare2005introduction}). Tärkeä ominaisuus tiivistefunktioissa on se, että ne toimivat yhteen suuntaan. Viestistä \emph{M} on helposti saatavissa tiiviste \emph{MD} funktiolla \emph{F}, mutta tiivisteestä \emph{MD} ja funktiosta \emph{F} ei voi helposti päätellä tai muodostaa viestiä \emph{M} (\cite{bakhtiari1995cryptographic}). Toisin sanoen kun jostain viestistä, esimerkiksi salasanasta, on tehty tiiviste niin ei ole olemassa yleistä menetelmää, jolla tämä \emph{hashaus} voitaisiin peruuttaa tai saataisiin selville itse funktio \emph{F}.
 
 Tiivistefunktiot ovat symmetrisen salauksen tapaan jo itsessään kvanttiturvallisia. Tiivistefunktiot muodostavat lohkosalauksen tapaan tietyn mittaisen bittijonon tiivisteen muodossa. Groverin algoritmia hyödyntäen on mahdollista tehdä hyökkäys tiivistefunktioita vastaan löytämällä törmäys (\emph{collision}) kahden tiivisteen välillä. Törmäyksellä tarkoitetaan tilannetta milloin kaksi eri syötettä antavat saman tiivisteen. Groverin algoritmilta voi suojautua pidentämällä tiivisteen pituutta. Turvallinen raja on, että \emph{n}-bitin suojaukseen tarvitaan 3\emph{n}-bittiä pitkä tiiviste. Esimerkiksi SHA-2 ja SHA-3 standardin tiivistefunktiot pitkillä bittijonoilla ovat kvanttiturvallisia. (\cite{mavroeidis2018impact}.)
 
 \section{Julkisen avaimen salaus}
 Epäsymmetrisessa eli julkisen avaimen salauksessa on käytössä kaksi avainta. Toinen on julkinen avain (\emph{public key}) ja toinen on salainen avain (\emph{secret key\footnote{Joskus myös \emph{private key}}}). Julkisen avaimen salausjärjestelmässä jokainen käyttäjä luo itselleen parin avaimia, jotka ovat salausavain ja purkuavain. Tässä tapauksessa purkuavain on salainen avain ja sitä ei missään nimessä pidä paljastaa muille. Tällöin salausavain on julkinen avain, joka saa ja pitää olla julkisesti esillä. (\cite{bellare2005introduction}.)
 
 Olkoon meillä lähettäjä \emph{S}, joka haluaa lähettää salatun viestin vastaanottajalle \emph{R}. Viestin lähetys toimii nyt seuraavasti:
 \begin{enumerate}
     \item \emph{S} salaa viestin \emph{R}:n julkisella avaimella $PK_{R}$.
     \item \emph{S} lähettää viestin vastaanottajalle \emph{R}.
     \item \emph{R} saa \emph{S}:n viestin vastaan.
     \item \emph{R} purkaa \emph{S}:n salauksen omalla salaisella avaimellaan $SK_{R}$.
 \end{enumerate}
Ero symmetriseen salaukseen epäsymmetrisessa salauksessa on siis se, että salaus puretaan eri avaimella kuin millä se salataan. (\cite{bellare2005introduction}.)
 
  Julkisen avaimen salausjärjestelmien tarjoama suojaus perustuu tiettyihin vaikeisiin lukuteorian ongelmiin (\cite{mavroeidis2018impact}). Monissa ongelmissa kyse on matemaattisista funktioista, jotka on helppo laskea yhteen suuntaan, mutta niiden peruuttaminen tai käänteisoperaatio taas on hyvin hankalaa. Näitä ongelmia ovat muun muassa suurten alkulukujen tekijöihinjako (\emph{Integer Factorization Problem}) sekä diskreetin logaritmin laskeminen (\emph{Discrete Logarithm Problem, DLP}).
  Yksi ensimmäisiä ja tänäkin päivänä käytetyimpiä julkisen avaimen salausjärjestelmiä on RSA, joka on nimetty keksijöidensä (Rivest, Shamir ja Adleman) mukaan (\cite{montgomery1994survey}). RSA:n tarjoama suojaus perustuu suurten alkulukujen tekijöihin jakoon. RSA:n avainpari luodaan seuraavasti  (\cite{10.1145/359340.359342}):
  
  \begin{enumerate}
  
      \item Valitaan kaksi suurta satunnaista alkulukua \emph{p} ja \emph{q}, siten että $p \neq q$. Nämä luvut $p$ ja $q$ on pidettävä salassa.
      
      \item Lasketaan $n = pq$. Tästä saatava $n$ saa olla julkisesti esillä.
      
      \item Valitaan jokin suuri satunnainen kokonaisluku $d$ siten että,
      \[syt(d, (p-1)(q-1)) = 1\]
      Tarkemmin sanottuna $d$ ja $(p-1)(q-1)$ ovat nyt keskenään suhteellisia alkulukuja (\emph{coprime} tai \emph{relatively prime}). Tämä tarkoittaa sitä, että niiden suurin yhteinen tekijä eli suurin kokonaisluku, joka jakaa sekä $d$:n että $(p-1)(q-1)$:n on 1.
      
      \item Lasketaan $e$, joka on $d$:n käänteisluku
      \[e d \equiv 1 \pmod{(p-1) (q-1)}\]
  \end{enumerate}
  Salausavain on nyt pari positiivisia kokonaislukuja ($e$, $n$). Purkuavain on pari positiivisia kokonaislukuja ($d$, $n$). (\cite{10.1145/359340.359342}..)
  
 Esitellään seuraavaksi diskreetin logaritmin ongelma. Olkoon meillä jokin äärellinen kunta $GF$($q$), jossa on $q$ alkiota (\cite{1055638}). Olkoon
 \begin{equation}
     \label{eq10}
     Y = \alpha^{X} \mod q, \; \text{jollekkin} \; 1 \leq X \leq q-1
 \end{equation}
 missä $\alpha$ on jokin määrätty alkio $GF$($q$):sta. Yhtälö voidaan muuttaa muotoon siten, että ratkaistaan $X$ 
 \begin{equation}
     \label{eq11}
     X = \log_{\alpha} Y \mod q, \; \text{jollekkin} \; 1 \leq Y \leq q-1
 \end{equation}
 Nyt siis $Y$:n selvittäminen $X$:stä on helppoa ja vaatii enimmillään $2 \times \log_{2}q$ kertolaskuoperaatiota. Toisaalta $X$:n selvittäminen $Y$:stä ei ole helppoa ja vaatii parhailla algoritmeilla $q^{1/2}$ kertolaskuoperaatiota. Esimerkiksi julkisen avaimen salausjärjestelmät kuten Diffie-Hellman ja elliptisen käyrän salaus (\emph{Elliptic Curve Cryptography, ECC}) hyödyntävät tätä (\cite{mavroeidis2018impact}).
 
 Edellä mainitut suurien kokonaislukujen tekijöihinjako ja diskreetin logaritmin löytäminen ovat ongelmia, joita on tutkittu paljon, mutta siitä huolimatta niille ei ole löytynyt polynomisessa ajassa toimivia algoritmeja (\cite{doi:10.1137/S0036144598347011}). Peter Shor esitteli vuonna 1994 Shorin algoritmin (\cite{mavroeidis2018impact}), jota käyttäen edellä mainitut vaikeat lukuteorian ongelmat voidaan ratkaista tehokkaasti eli polynomisessa ajassa kvanttitietokoneella (\cite{doi:10.1137/S0036144598347011}). Käsittelemme Shorin algoritmia myöhemmin kappaleessa 3.2 kvanttilaskennan yhteydessä.
 
 \section{Digitaalinen allekirjoitus}
 Digitaalinen allekirjoitus (\emph{digital signature}) on menetelmä, jolla voidaan todentaa digitaalisia dokumentteja. Sen toimintaperiaate on analoginen perinteisiin allekirjoituksiin verrattuna. Digitaalisella allekirjoituksella todennetaan muun muassa digitaalisen dokumentin datan eheys. Eheydellä voidaan tässä kontekstissa tarkoittaa esimerkiksi dokumentin päivämäärää tai sisältämää informaatiota. (\cite{1649003}.)
 
 Digitaalinen allekirjoitus on käytännössä bittijono, jonka jokin digitaalisen allekirjoituksen luomiseen tarkoitettu algoritmi luo. Tämä allekirjoitus muodostetaan allekirjoitettavien dokumenttien ja jonkin vain viestin lähettäjän määrittelemän tiedon mukaan. Digitaalisen allekirjoituksen luominen koostuu yleensä kolmesta eri vaiheesta; salausavaimen generointi, digitaalisen allekirjoituksen generointi ja digitaalisen allekirjoituksen varmennus. (\cite{1649003}.)
 
 Edellä mainittua julkisen avaimen salausta voidaan käyttää digitaalisen allekirjoituksen luomiseen (\cite{mavroeidis2018impact}). Esimerkiksi dokumentin allekirjoittaja \emph{A} voi allekirjoittaa dokumentin omalla salaisella avaimellaan $SK_{A}$. Dokumentin varmentaja \emph{V} voi siten todentaa \emph{A}:n allekirjoituksen \emph{A}:n julkisella avaimella $PK_{A}$. Kaksi suosituinta ja eniten käytettyä julkisen avaimen salaukseen perustuvaa digitaalisen allekirjoittamisen menetelmää ovat RSA\footnote{Tässä RSA:lla tarkoitetaan samannimistä digitaalisen allekirjoituksen menetelmää, joka hyödyntää kappaleessa 2.3 mainittua RSA-algoritmia.} ja DSA (\emph{Digital Signature Algorithm}) (\cite{1649003}). Koska RSA:n turvallisuus perustuu suurten alkulukujen tekijöihinjakoon ja DSA:n turvallisuus diskreettien logaritmien laskemiseen, niin Shorin algoritmi uhkaa myös näitä digitaalisen allekirjoituksen menetelmiä.