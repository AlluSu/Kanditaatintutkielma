\chapter{Kvanttiturvallinen salaus\label{results}}
Seuraavaksi käymme yleisesti läpi muutamia kvanttiturvallisia salausjärjestelmiä ja menetelmiä.

\section{Kvanttiturvalliset salausjärjestelmät}

\subsection{Tiiviste-pohjainen salaus}
Tiiviste-pohjainen salaus (\emph{hash-based cryptography}) on salausjärjestelmä, joka pohjautuu aiemmin mainittuihin tiivistefunktioihin. Tiiviste-pohjaista salausta voidaan käyttää kvanttiturvallisen digitaalisen allekirjoituksen luomiseen (\emph{hash-based digital signature}). Idea tiiviste-pohjaisen allekirjoituksen luomiseen tulee Leslie Lamportin artikkelista vuodelta 1979.

\subsection{Koodi-pohjainen salaus}
Koodi-pohjainen salaus (\emph{code-based cryptography}) perustuu Robert McEliecen artikkeliin \emph{A Public-Key Cryptosystem Based On Algebraic Coding Theory} vuodelta 1978. McEliecen esittelemässä salauksessa salattavaa viestiä toistetaan monta kertaa putkeen, jonka jälkeen siihen lisätään tahallisesti virheitä. Näitä tahallaan lisättyjä virheitä kutsutaan virheenkorjauskoodeiksi (\emph{error-correction codes}).

\subsection{Hila-pohjainen salaus}
Hila-pohjainen salaus (\emph{lattice-based cryptography}) on kvanttiturvallinen salauksen muoto. Hila-pohjaisessa salauksessa salaus perustuu tiettyihin laskennallisesti vaativiin ongelmiin koskien hiloja. Hila on matemaattinen objekti, joka kuvataan joukkona pisteitä \emph{n}-ulotteisessa avaruudessa ja sillä on jonkinlainen jaksollinen rakenne.

\subsection{Monimuuttuja-pohjainen salaus}
Monimuuttuja-pohjainen (\emph{multivariate-based cryptography}) on eräs kvanttiturvallisen salauksen muoto. Monimuuttuja-pohjainen salaus perustuu monta muuttujaa sisältävien yhtälöryhmien ratkaisun vaikeuteen. Yhtälöryhmien yhtälöt ovat neliöllisiä eli epälineaarisia äärellisessä kunnassa. Tällaisen yhtälöryhmän ratkaisu on NP-kova ongelma ratkaista eli se on laskennallisesti hyvin vaativa ongelma.