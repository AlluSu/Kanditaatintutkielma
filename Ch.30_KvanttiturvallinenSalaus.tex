\chapter{Kvanttiturvallinen salaus\label{results}}

\section{Kvanttiturvalliset salausjärjestelmät}

Seuraavaksi tutustumme muutamiin kvanttiturvallisiin salausjärjestelmiin. Mainittujen lisäksi on olemassa myös lähteistä riippuen muita kvanttiturvallisia salausjärjestelmiä, mutta emme kuitenkaan tämän tutkielman laajuuden puitteissa perehdy aivan kaikkiin. Esimerkiksi emme käsittele elliptisiin käyriin perustuvia kvanttiturvallisia salausjärjestelmiä (\emph{Supersingular elliptic curve isogeny cryptography} tai joissain lähteissä \emph{isogeny-based cryptography}.)

%Viitteitä?

Käsittelemme kvanttiturvallisista salausjärjestelmistä niiden idean sekä yleisiä ominaisuuksia esimerkiksi suorituskyvyssä ja julkisen avaimen koossa. Näistä eri kvanttiturvallisista salausjärjestelmistä on olemassa ja implementoitu useita eri algoritmeja ja variantteja, joilla saadaan esimerkiksi parempi suorituskyky, mutta isompi julkisen avaimen koko tai toisin päin. Kuitenkin kaikkien niiden toiminta pohjautuu tässä esitellyihin ideoihin.

Tällä hetkellä tutkitaan ja kehitetään useita eri kvanttiturvallisia salausjärjestelmiä. \emph{National Institute of Standards and Technology (NIST)} ylläpitää prosessia, jossa etsitään kvanttiturvallisia julkisen avaimen salausjärjestelmiä joista tulisi uusi kvanttiturvallinen standardi. NIST etsii ja tutkii esimerkiksi vaihtoehtoisia salausjärjestelmiä joilla voitaisiin korvata esimerkiksi RSA kvanttitietokoneiden yleistyessä (\cite{alagic2020status}).

\subsection{Tiivistepohjainen salaus}
Tiivistepohjainen salaus (\emph{hash-based cryptography}) on salausjärjestelmä, joka pohjautuu aiemmin mainittuihin tiivistefunktioihin. Tiivistepohjaista salausta voidaan käyttää kvanttiturvallisen digitaalisen allekirjoituksen luomiseen (\emph{hash-based digital signature}). Tiivistepohjaiset salausjärjestelmät ovat turvallisia kvanttitietokoneita vastaan niiden luonteen ja rakenteen takia. 

\subsection{Koodipohjainen salaus}
Koodipohjainen salaus (\emph{code-based cryptography}) perustuu Robert McEliecen artikkeliin \emph{A Public-Key Cryptosystem Based On Algebraic Coding Theory} vuodelta 1978 (\cite{8012331}). McEliecen esittelemässä salauksessa salattavaa viestiä toistetaan monta kertaa putkeen, jonka jälkeen siihen lisätään tahallisesti virheitä. Näitä tahallaan lisättyjä virheitä kutsutaan virheenkorjauskoodeiksi (\emph{error-correction codes}). Julkisena avaimena tässä salausjärjestelmässä toimii niin kutsuttu kehitinmatriisi (\emph{generator matrix}). Salausjärjestelmän turvallisuus perustuu siihen, että tällaisen salauksen purkaminen on yleisesti ottaen vaikeaa. Kuinka löytää salattu viesti kaiken muun virheellisen tiedon joukosta jos ei ole tietoa miten se on salattu? Tämä on NP-täydellinen ongelma. Lisäksi kehitinmatriisin erottaminen satunnaisesta matriisista on hankalaa.

McEliecen esittämä salausjärjestelmä on hyvä kandidaatti kvanttiturvalliseksi julkisen avaimen salausjärjestelmäksi. Hyvinä puolina salausjärjestelmässä on muun muassa laskennallinen tehokkuus viestiä salattaessa ja salausta purettaessa. Salausjärjestelmän tuoma turva on hyvin ymmärretty ja kestänyt tarkastelua yli 40 vuotta.

Huonoina puolina mainittakkoon, että digitaalisten allekirjoitusten tekemiseen se ei sovellu. Lisäksi se tuottaa kooltaan hyvin suuria julkisia avaimia jos käytetään kvanttiturvallista versiota. Tällöin puhutaan noin 1 megatavun kokoisista avaimista mikä saattaa mahdollisesti rajoittaa salausjärjestelmän käyttöä esimerkiksi sulautetuissa järjestelmissä tai muissa järjestelmissä, jossa muistia on vähän ja resursseja rajoitetusti. Kuitenkin McEliecen salausjärjestelmästä on olemassa variantteja, joissa on mahdollistettu pienempi avaimen koko.


\subsection{Hilapohjainen salaus}
Hilapohjainen salaus (\emph{lattice-based cryptography}) on kvanttiturvallinen salauksen muoto (\cite{regev2006lattice}). Hilapohjaisessa salauksessa salaus perustuu tiettyihin laskennallisesti vaativiin ongelmiin koskien hiloja. Hila on matemaattinen objekti, joka kuvataan joukkona pisteitä \emph{n}-ulotteisessa avaruudessa ja sillä on jonkinlainen jaksollinen rakenne. Formaalimmin hila on \emph{n}:stä lineaarisesti riippumattomasta vektorista muodostuva joukko vektoreita: 

\[L(v_1,...v_n) =\Bigg\{ {\sum_{i=1}^{n}\alpha_i 
v_i|\alpha_i \in \mathbb{Z}} \Bigg\}\]. 

Missä vektorit $v_1,...,v_n$ ovat hilan kanta.

Monissa hiloihin liittyvissä 
laskennallisissa ongelmissa ollaan kiinnostuneita löytämään jokin lyhin vektori tai vektoreita. Lyhimmän vektorin ongelma (\emph{SVP, Shortest Vector Problem)} on yleinen esimerkki laskennallisesta ongelmasta kun puhutaan hiloista. \emph{SVP}:ssä annetaan hilan kanta ja tehtävän on määritellä lyhin vektori hilassa, joka ei ole nollavektori. \emph{SVP}:n haastavuus piilee siinä, että hilalla on monta eri kantaa ja hilan kanta koostuu erittäin pitkistä vektoreista.

Hilapohjaisessa salauksessa huomattavaa on se miten salauksen vahvuus perustuu ongelmien pahimman tapauksen mukaan (\emph{worst-case hardness}). Esimerkiksi nykyisissä salausjärjestelmissä käyettävää suurten alkulukujen tekijöihin jakamisen vaikeutta kuvataan keskimääräisen tapauksen mukaan (\emph{average-case hardness}). Tämä tarkoittaa käytännössä sitä, että alkulukujen tekijöihin jako on helppo laskennallinen ongelma, jos tekijät ovat esimerkiksi pieniä tai parillisia lukuja. Tekijöihin jako taas vaikeutuu, jos luvut ovat hyvin suuria. Tällaista ilmiöitä ei esiinny hilojen kanssa, että olisi jokin tietty jakauma mikä määrittelee onko ongelma laskennallisesti helppo vai vaikea.

Aluksi hilapohjaiset salausjärjestelmät olivat hyvin epäkäytännöllisiä niiden tehokkuuden ja suurten julkisten avainten kokojen takia. Tutkimuksissa on kuitenkin edistytty ja uranuurtaja tässä oli Regevin esittelemä variantti salausjärjestelmästä nimeltään \emph{LWE} (\emph{Learning With Errors}). Myöhemmin esiteltiin vielä optimoidumpi versio \emph{Ring-LWE} (\cite{8275352}).

Hilapohjaisen salauksen hyviä puolia ovat muun muassa salaukseen tarvittavan laskennan yksinkertaisuus, koska ne vaativat usein vain modulaariartimetiikkaa. Tämä on eduksi silloin jos resursseja kuten muistia ja laskentatehoa on vain rajoitetusti kuten vaikka sulautetuissa järjestelmissä.

Huonona puolena hilapohjaisessa salauksessa on se, että salatun viestin koko kasvaa noin 30-kertaiseksi verrattuna alkuperäiseen salaamattomaan viestiin. Esimerkiksi edellä mainittu \emph{Ring-LWE} kasvattaa 128-tavuisen salaamattoman viestin 4096-tavuiseksi. Nykyisissä julkisen avaimen salausjärjestelmissä salaamattoman ja salatun viestin koko on täysin sama (\cite{8275352}).

Hiloihin perustuviin ongelmiin ei ole tällä hetkellä tiedossa olevia kvanttialgoritmeja mikä tekee siitä luonnollisesti oivallisen kvanttiturvallisen salausmenetelmän (\cite{regev2006lattice}).

\subsection{Monimuuttujapohjainen salaus}
Monimuuttujapohjainen salaus (\emph{multivariate-based cryptography}) on eräs kvanttiturvallisen salauksen muoto. Monimuuttujapohjaista salausta käytetään monimuuttujapohjaisen julkisen avaimen salausjärjestelmän muodostamiseen (\emph{Multivariate public key cryptosystem}) (\cite{Ding2009}). Monimuuttujapohjainen salaus perustuu monta muuttujaa sisältävien yhtälöryhmien ratkaisun vaikeuteen. Yhtälöryhmien yhtälöt ovat neliöllisiä eli epälineaarisia äärellisessä kunnassa.

Monimuuttujapohjaisessa julkisen avaimen salausjärjestelmässä julkinen avain on yhtälöryhmä neliöllisiä polynomeja, joissa on monta muuttujaa pienessä äärellisessä kunnassa \emph{K}. Esimerkkinä (\cite{8012305}): 

    \begin{align*}
        p^{(1)}(x_{1}...,x_{n}) &= \sum_{i=1}^{n} p_{ij}^{(1)}x_{i}x_ {j}+\sum_{i=1}^{n}p_ {i}^{(1)}x_{i}+p_ {0}^{(1)} \\
        p^{(2)}(x_{1}...,x_{n}) &= \sum_{i=1}^{n} p_{ij}^{(2)}x_{i}x_ {j}+\sum_{i=1}^{n}p_ {i}^{(2)}x_{i}+p_ {0}^{(2)} \\
        ..., \\
        p^{(m)}(x_{1}...,x_{n}) &= \sum_{i=1}^{n} p_{ij}^{(m)}x_{i}x_ {j}+\sum_{i=1}^{n}p_ {i}^{(m)}x_{i}+p_ {0}^{(m)}
    \end{align*}
    
Tällaisen julkisen avaimen salausjärjestelmän turvallisuus perustuu niin kutsuttuun \emph{MQ}-ongelmaan (\emph{Multivariate quadratic polynomial problem}).

\emph{MQ}-ongelma: Ratkaise yhtälöryhmä $p_{1}(x) = p_{2}(x) = ... = p_ {m}(x) = 0$, missä jokainen $p_{i}$ on neliöllinen joukossa $x = (x_{1}, ..., x_{n})$ (\cite{Ding2009}).

\emph{MQ}-ongelma on todistettu NP-kovaksi ongelmaksi jokaisessa kunnassa ja sen uskotaan siten olevan vaikea ongelma myös kvanttitietokoneille. Tähän perustuu monimuuttujapohjaisen julkisen avaimen salausjärjestelmän suojaus (\cite{8012305}).

Monimuuttujapohjaisessa salausjärjestelmässä haitallista on niiden tuottamien avaimien suuret koot (\cite{Ding2009}). Keskimäärin monimuuttujapohjaiset salausjärjestelmät luovat julkisen avaimen jonka koko on väliltä 10-100 kilotavua (\cite{8012305}). Esimerkiksi monimuuttujapohjainen julkisen avaimen salausjärjestelmä jossa on \emph{m} polynomia ja \emph{n} muuttujaa tuottaa julkisen avaimen jossa on $m(n+2)(n+1)/2$ termiä. \emph{Rainbow}, joka on monimuuttujapohjainen allekirjoittamiseen luotu järjestelmä, tuottaa julkisen avaimen joka on kooltaan yli 16 kilotavua. Nämä suuret avaintenkoot saattavat tuottaa ongelmia muun muassa sulautetuissa järjestelmissä tai järjestelmissä, joissa verkkoyhteydet ovat hyvin rajoittuneet ja jos avaimia pitäisi lähettää monta kertaa verkon yli.

\emph{Rainbow} on yksi NIST:in tutkima ehdokas, josta voisi tulla uusi kvanttiturvallinen standardi allekirjoitusten tekoon. Tutkimusalueen alalla tulevaisuuden tutkimussuunnitelmiin kuuluu julkisen avaimen pienentämisen tutkiminen. Monimuuttujapohjaisen salauksen parissa on salausmenetelmiä, joiden kvanttiturvallisuutta ei ole rigöörisesti todistettu ja sieltä puuttuu formaalia teoriapohjaa (\cite{8012305}). 

Monimuuttujapohjaisessa salausjärjestelmässä on myös paljon etuja. Monimuuttujapohjaiset salausjärjestelmät voidaan implementoida tehokkaiksi jolloin ne ovat nopeampia kuin muut kvanttiturvallisest salausjärjestelmät. Monimuuttujapohjaiset salausjärjestelmät vaativat kohtuullisesti laskentatehoa, jolloin sovelluskohteita voi löytyä IoT:n parista. Ne käyttävät ainoastaan aritmeettisia laskuoperaatioita kuten kertolaskua ja yhteenlaskua. Tällöin niitä voi hyödyntää laitteissa, joissa ei ole paljoa resursseja saatavilla, kuten esimerkiksi älykorteissa ja RFID-tunnisteissa. Monimuuttujapohjaiset allekirjoitukset ovat pituudeltaan hyvin lyhyitä verrattuna esimerkiksi klassisiin tai kvanttiturvallisiin allekirjoituksiin.