\chapter{Kvanttiturvallinen salaus\label{results}}
Seuraavaksi käymme läpi muutamia kvanttiturvallisia salausjärjestelmiä.

\section{Kvanttiturvalliset salausjärjestelmät}

\subsection{Tiiviste-pohjainen salaus}
Hash-based cryptography

\subsection{Koodi-pohjainen salaus}
Koodi-pohjainen salaus (englanniksi code-based cryptography) perustuu Robert McEliecen artikkeliin "A Public-Key Cryptosystem Based On Algebraic Coding Theory" vuodelta 1978. Koodi-pohjaisessa salauksessa salateksti saadaan siten, että tekstiin lisätään tahallaan virheitä.

\subsection{Hila-pohjainen salaus}
Hila-pohjainen salaus (englanniksi lattice-based cryptography) on kvanttiturvallinen salauksen muoto. Hila-pohjaisessa salauksessa salaus perustuu tiettyihin laskennallisesti vaativiin ongelmiin koskien hiloja. Hila on joukko pisteitä n-ulotteisessa avaruudessa, jolla on jonkinlainen jaksollinen rakenne.

\subsection{Monimuuttuja-pohjainen salaus}
Monimuuttuja-pohjainen (englanniksi multivariate-based cryptography) on eräs kvanttiturvallisen salauksen muoto. Monimuuttuja-pohjainen salaus perustuu monta muuttujaa sisältävien yhtälöryhmien ratkaisun vaikeuteen. Yhtälöryhmien yhtälöt ovat neliöllisiä eli epälineaarisia äärellissä kunnassa joka on NP-kova ongelma ratkaista. 