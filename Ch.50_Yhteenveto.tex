\chapter{Yhteenveto\label{conclusions}}
% Puhu mitä työssä on tehty!

Tutkielmassa tutustuimme pintapuolisesti nykyisiin salausjärjestelmiin ja niiden kvanttiturvallisuuten, kvanttilaskentaan ja Shorin algoritmiin sekä kvanttiturvallisiin salausjärjestelmiin, joihin Shorin algoritmia ei voi soveltaa ja jotka perustuvat kvanttitietokoneellekkin vaikeisiin ongelmiin.

Kvanttitietokoneiden kehittyessä kvanttiturvallinen salaus on tullut ajankohtaiseksi. Kvanttitietokoneiden mahdollistama eksponentiaalinen laskentateho klassisiin tietokoneisiin verrattuna uhkaa nykyisiä julkisen avaimen salausjärjestelmiä. Etenkin kvanttirinnakkaisuutta hyödyntävä Shorin algoritmi ratkaisee klassiselle tietokoneelle vaikeat laskennalliset ongelmat tehokkaasti. Esimerkiksi yksi ensimmäisiä ja käytetyimpiä julkisen avaimen salausjärjestelmiä, RSA, käyttää suurten kokonaislukujen tekijöihinjakoa salauksessaan minkä Shorin algoritmi pystyy murtamaan tehokkaasti.

Tutkielmassa keskityimme kolmeen eri kvanttiturvalliseen salausmenetelmään; koodipohjaiseen, hilapohjaiseen ja monimuuttujapohjaiseen salaukseen. Jokainen salaus perustui omaan NP-täydelliseen ongelmaan; koodipohjaisella  \emph{Generic Decoding Problem}, hilapohjaisella \emph{Shortest Vector Problem} sekä muut hiloihin liittyvät laskennalliset ongelmat ja monimuuttujapohjaisella \emph{MQ}-ongelma. Jokaisessa salauksessa oli omat ominaisuutensa ja kehityskohteensa. Jokainen salausmenetelmä todettiin laskennallisesti tehokkaaksi ja nopeaksi, mutta monimuuttujapohjainen salaus todettiin kaikista kolmesta nopeimmaksi. Koodipohjaisesta ja monimuuttujapohjaisesta salauksesta todettiin huonona puolena suuret julkiset avaimet, jotka rajoittavat näiden salausjärjestelmien käyttöä esimerkiksi sulautetuissa järjestelmissä. Lisäksi koodipohjaisessa ja monimuuttujapohjaisessa oli puutteita kvanttiturvallisuuden teoriapohjassa, vaikka käytännössä ovat pysyneet kvanttiturvallisina. Hilapohjaisesta salauksesta todettiin kehityskohteena salatun viestin koon kasvaminen monikymmenkertaiseksi. Esittelimme myös jokaista salausmenetelmää edustavan NIST:in standardointikierroksen finalistin. Finaalissa ovat tällä hetkellä julkisen avaimen salausmenetelmiä edustamassa koodipohjainen \emph{Classic McEliece} ja hilapohjainen \emph{NTRU}. Allekirjoittamiseen luotua järjestelmää edustaa monimuuttujapohjainen \emph{Rainbow}.

Vaikka suuren kokoluokan kvanttitietokoneet eivät juuri tällä hetkellä ole suuri uhka tietoturvalle niin kvanttiturvalliseen salaukseen pitäisi varautua jo nyt. Retrospektiivisellä purkamisella tarkoitetaan hyökkäystä, missä hyökkääjä kerää salattua tietoa talteen tulevaisuutta varten ja purkaa salatun tiedon jälkeenpäin kun löydetään keino purkaa salaus.