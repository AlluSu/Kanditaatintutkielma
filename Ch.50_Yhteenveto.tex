\chapter{Yhteenveto\label{conclusions}}
Kvanttitietokoneiden kehitys on vielä hieman lapsenkengissä. Moni voi ajatella, että kvanttitietokoneilla tehdyiltä hyökkäyksiltä ei tarvitse suojautua. Kvanttitietokoneet ja kvanttilaskenta eivät ole kovin arkipäiväisiä asioita ja saattavat tuntua hyvin abstrakteilta ja oudoilta konsepteilta.

Kuten aikaisemmin mainittiin niin nykypäivän kvanttitietokoneiden rakentaminen ja operoiminen ei ole kovin yksinkertaista eikä halpaa. Nykypäivän kvanttitietokoneet tarvitsevat hyvin paljon infrastruktuuria ympärilleen ja vievät paljon tilaa. Lisäksi kvanttilaskenta ja siinä käytettävien kubittien hallinta on hankalaa. Tilanne on kuitenkin hyvin analoginen jos sitä vertaa nykyisen tietojenkäsittelytieteen historiaan. Tietojenkäsittelytieteen alkuaikoina tietokoneet olivat kalliita, isokokoisia, melko tehottomia ja hajoilivat herkästi. Ajan myötä kuitenkin tietokoneet saatiin puristettua paljon pienempään kokoon sekä paljon kestävimmiksi ja tehokkammiksi. Todennäköisesti myös jossain kohtaa tulevaisuudessa kvanttitietokoneet saadaan paljon kooltaan pienemmiksi ja toiminnaltaan luotettavimmiksi.

Vaikka kvanttitietokoneet kuulostavat hyvin utopistiselta niin niillä tehtyihin hyökkäyksiin tulisi varautua jo nyt eikä vasta joskus tulevaisuudessa. Joku pahantahtoinen taho voisi esimerkiksi varastaa salattuja tietoja juuri tällä hetkellä ja odottaa rauhassa suuren koko luokan kvanttitietokoneiden yleistymistä ja kehittymistä. Tällöin nämä ns. klassisilla menetelmillä suojatut tiedot voitaisiin purkaa kvanttitietokoneen avulla helposti ja nopeasti.