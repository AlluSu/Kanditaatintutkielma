\chapter{Yhteenveto\label{conclusions}}
% Puhu mitä työssä on tehty!

Tutkielmassa tutustuimme pintapuolisesti nykyisiin salausjärjestelmiin ja niiden kvanttiturvallisuuten, kvanttilaskentaan ja Shorin algoritmiin sekä kvanttiturvallisiin salausjärjestelmiin, joihin Shorin algoritmia ei voi soveltaa.

Kvanttitietokoneiden kehittyessä kvanttiturvallinen salaus on tullut ajankohtaiseksi. Kvanttitietokoneiden mahdollistama eksponentiaalinen laskentateho klassisiin tietokoneisiin verrattuna uhkaa nykyisiä julkisen avaimen salausjärjestelmiä. Etenkin kvanttirinnakkaisuutta hyödyntävä Shorin algoritmi ratkaisee klassiselle tietokoneelle vaikeat laskennalliset ongelmat helposti. Esimerkiksi yksi ensimmäisiä ja käytetyimpiä julkisen avaimen salausjärjestelmiä, RSA, käyttää suurten kokonaislukujen tekijöihinjakoa salauksessaan minkä Shorin algoritmi voi murtaa tehokkaasti.

Tutkielmassa keskityimme kolmeen eri kvanttiturvalliseen salausmenetelmään; koodipohjaiseen, hilapohjaiseen ja monimuuttujapohjaiseen salaukseen. Jokaisessa menetelmässä oli omat ominaisuutensa ja kehityskohteensa. Jokainen salausmenetelmä todettiin laskennallisesti tehokkaaksi ja nopeaksi, mutta monimuuttujapohjainen salaus todettiin kaikista kolmesta nopeimmaksi. Koodipohjaisesta ja monimuuttujapohjaisesta salauksesta todettiin huonona puolena suuret julkiset avaimet, jotka rajoittavat näiden salausjärjestelmien käyttöä esimerkiksi sulautetuissa järjestelmissä. Hilapohjaisesta salauksesta todettiin kehityskohteena salatun viestin koon kasvaminen monikymmenkertaiseksi. Koodipohjaisesta ja hilapohjaisesta salauksesta todettiin, että niihin ei ole löydetty kvanttialgoritmeja, jotka tekisivät ne turvattomiksi.

Vaikka suuren kokoluokan kvanttitietokoneet eivät juuri tällä hetkellä ole suuri uhka tietoturvalle niin kvanttiturvalliseen salaukseen pitäisi varautua jo nyt. Retrospektiivisellä purkamisella tarkoitetaan hyökkäystä, missä hyökkääjä kerää salattua tietoa talteen tulevaisuutta varten ja purkaa salatun tiedon jälkeenpäin kun löydetään keino purkaa salaus.

Tulevaisuudessa odotetaan, että kvanttiturvallisia salausmenetelmiä saadaan kehitettyä eteenpäin. National Institute of Standards and Technology (NIST) suorittaa kvanttiturvallisten julkisen avaimen salausjärjestelmien standardointia. Tällä hetkellä on käynnissä kilpailun kolmas kierros, johon on valittu seitsemän finalistia ja joista valitaan uusi julkisen avaimen salauksen kvanttiturvallinen standardi.