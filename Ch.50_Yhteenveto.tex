\chapter{Yhteenveto\label{conclusions}}
Puhu mitä työssä on tehty!

Tässä tutkielmassa tutustuimme nykyisiin salausjärjestelmiin, kvanttilaskentaan ja Shorin algoritmiin sekä kvanttiturvallisiin salausjärjestelmiin, jotka pystyvät vastustamaan Shorin algoritmia.

Ensiksi tutustuimme nykyisiin salausjärjestelmiin, joihin tietokonetta ja internetiä käyttäessä törmää nykyään väistämättä. Käsittelimme symmetrisen salauksen, tiivistefunktiot, epäsymmetrisen eli julkisen avaimen salauksen sekä digitaaliset allekirjoitukset.